% Options for packages loaded elsewhere
\PassOptionsToPackage{unicode}{hyperref}
\PassOptionsToPackage{hyphens}{url}
%
\documentclass[
]{book}
\usepackage{lmodern}
\usepackage{amssymb,amsmath}
\usepackage{ifxetex,ifluatex}
\ifnum 0\ifxetex 1\fi\ifluatex 1\fi=0 % if pdftex
  \usepackage[T1]{fontenc}
  \usepackage[utf8]{inputenc}
  \usepackage{textcomp} % provide euro and other symbols
\else % if luatex or xetex
  \usepackage{unicode-math}
  \defaultfontfeatures{Scale=MatchLowercase}
  \defaultfontfeatures[\rmfamily]{Ligatures=TeX,Scale=1}
\fi
% Use upquote if available, for straight quotes in verbatim environments
\IfFileExists{upquote.sty}{\usepackage{upquote}}{}
\IfFileExists{microtype.sty}{% use microtype if available
  \usepackage[]{microtype}
  \UseMicrotypeSet[protrusion]{basicmath} % disable protrusion for tt fonts
}{}
\makeatletter
\@ifundefined{KOMAClassName}{% if non-KOMA class
  \IfFileExists{parskip.sty}{%
    \usepackage{parskip}
  }{% else
    \setlength{\parindent}{0pt}
    \setlength{\parskip}{6pt plus 2pt minus 1pt}}
}{% if KOMA class
  \KOMAoptions{parskip=half}}
\makeatother
\usepackage{xcolor}
\IfFileExists{xurl.sty}{\usepackage{xurl}}{} % add URL line breaks if available
\IfFileExists{bookmark.sty}{\usepackage{bookmark}}{\usepackage{hyperref}}
\hypersetup{
  pdftitle={R语言笔记},
  pdfauthor={云腾足下},
  hidelinks,
  pdfcreator={LaTeX via pandoc}}
\urlstyle{same} % disable monospaced font for URLs
\usepackage{color}
\usepackage{fancyvrb}
\newcommand{\VerbBar}{|}
\newcommand{\VERB}{\Verb[commandchars=\\\{\}]}
\DefineVerbatimEnvironment{Highlighting}{Verbatim}{commandchars=\\\{\}}
% Add ',fontsize=\small' for more characters per line
\usepackage{framed}
\definecolor{shadecolor}{RGB}{248,248,248}
\newenvironment{Shaded}{\begin{snugshade}}{\end{snugshade}}
\newcommand{\AlertTok}[1]{\textcolor[rgb]{0.94,0.16,0.16}{#1}}
\newcommand{\AnnotationTok}[1]{\textcolor[rgb]{0.56,0.35,0.01}{\textbf{\textit{#1}}}}
\newcommand{\AttributeTok}[1]{\textcolor[rgb]{0.77,0.63,0.00}{#1}}
\newcommand{\BaseNTok}[1]{\textcolor[rgb]{0.00,0.00,0.81}{#1}}
\newcommand{\BuiltInTok}[1]{#1}
\newcommand{\CharTok}[1]{\textcolor[rgb]{0.31,0.60,0.02}{#1}}
\newcommand{\CommentTok}[1]{\textcolor[rgb]{0.56,0.35,0.01}{\textit{#1}}}
\newcommand{\CommentVarTok}[1]{\textcolor[rgb]{0.56,0.35,0.01}{\textbf{\textit{#1}}}}
\newcommand{\ConstantTok}[1]{\textcolor[rgb]{0.00,0.00,0.00}{#1}}
\newcommand{\ControlFlowTok}[1]{\textcolor[rgb]{0.13,0.29,0.53}{\textbf{#1}}}
\newcommand{\DataTypeTok}[1]{\textcolor[rgb]{0.13,0.29,0.53}{#1}}
\newcommand{\DecValTok}[1]{\textcolor[rgb]{0.00,0.00,0.81}{#1}}
\newcommand{\DocumentationTok}[1]{\textcolor[rgb]{0.56,0.35,0.01}{\textbf{\textit{#1}}}}
\newcommand{\ErrorTok}[1]{\textcolor[rgb]{0.64,0.00,0.00}{\textbf{#1}}}
\newcommand{\ExtensionTok}[1]{#1}
\newcommand{\FloatTok}[1]{\textcolor[rgb]{0.00,0.00,0.81}{#1}}
\newcommand{\FunctionTok}[1]{\textcolor[rgb]{0.00,0.00,0.00}{#1}}
\newcommand{\ImportTok}[1]{#1}
\newcommand{\InformationTok}[1]{\textcolor[rgb]{0.56,0.35,0.01}{\textbf{\textit{#1}}}}
\newcommand{\KeywordTok}[1]{\textcolor[rgb]{0.13,0.29,0.53}{\textbf{#1}}}
\newcommand{\NormalTok}[1]{#1}
\newcommand{\OperatorTok}[1]{\textcolor[rgb]{0.81,0.36,0.00}{\textbf{#1}}}
\newcommand{\OtherTok}[1]{\textcolor[rgb]{0.56,0.35,0.01}{#1}}
\newcommand{\PreprocessorTok}[1]{\textcolor[rgb]{0.56,0.35,0.01}{\textit{#1}}}
\newcommand{\RegionMarkerTok}[1]{#1}
\newcommand{\SpecialCharTok}[1]{\textcolor[rgb]{0.00,0.00,0.00}{#1}}
\newcommand{\SpecialStringTok}[1]{\textcolor[rgb]{0.31,0.60,0.02}{#1}}
\newcommand{\StringTok}[1]{\textcolor[rgb]{0.31,0.60,0.02}{#1}}
\newcommand{\VariableTok}[1]{\textcolor[rgb]{0.00,0.00,0.00}{#1}}
\newcommand{\VerbatimStringTok}[1]{\textcolor[rgb]{0.31,0.60,0.02}{#1}}
\newcommand{\WarningTok}[1]{\textcolor[rgb]{0.56,0.35,0.01}{\textbf{\textit{#1}}}}
\usepackage{longtable,booktabs}
% Correct order of tables after \paragraph or \subparagraph
\usepackage{etoolbox}
\makeatletter
\patchcmd\longtable{\par}{\if@noskipsec\mbox{}\fi\par}{}{}
\makeatother
% Allow footnotes in longtable head/foot
\IfFileExists{footnotehyper.sty}{\usepackage{footnotehyper}}{\usepackage{footnote}}
\makesavenoteenv{longtable}
\usepackage{graphicx,grffile}
\makeatletter
\def\maxwidth{\ifdim\Gin@nat@width>\linewidth\linewidth\else\Gin@nat@width\fi}
\def\maxheight{\ifdim\Gin@nat@height>\textheight\textheight\else\Gin@nat@height\fi}
\makeatother
% Scale images if necessary, so that they will not overflow the page
% margins by default, and it is still possible to overwrite the defaults
% using explicit options in \includegraphics[width, height, ...]{}
\setkeys{Gin}{width=\maxwidth,height=\maxheight,keepaspectratio}
% Set default figure placement to htbp
\makeatletter
\def\fps@figure{htbp}
\makeatother
\setlength{\emergencystretch}{3em} % prevent overfull lines
\providecommand{\tightlist}{%
  \setlength{\itemsep}{0pt}\setlength{\parskip}{0pt}}
\setcounter{secnumdepth}{5}
\usepackage{ctex}

%\usepackage{xltxtra} % XeLaTeX的一些额外符号
% 设置中文字体
%\setCJKmainfont[BoldFont={黑体},ItalicFont={楷体}]{新宋体}

% 设置边距
\usepackage{geometry}
\geometry{%
  left=2.0cm, right=2.0cm, top=3.5cm, bottom=2.5cm} 

\usepackage{amsthm,mathrsfs}
\usepackage{booktabs}
\usepackage{longtable}
\makeatletter
\def\thm@space@setup{%
  \thm@preskip=8pt plus 2pt minus 4pt
  \thm@postskip=\thm@preskip
}
\makeatother
\usepackage[]{natbib}
\bibliographystyle{apalike}

\title{R语言笔记}
\author{云腾足下}
\date{2020年5月}

\begin{document}
\maketitle

{
\setcounter{tocdepth}{1}
\tableofcontents
}
\hypertarget{ux5e8f}{%
\chapter*{序}\label{ux5e8f}}
\addcontentsline{toc}{chapter}{序}

不知道说啥,还是留首诗吧。

\begin{quote}
赵客缦胡缨,吴钩霜雪明。

银鞍照白马,飒沓如流星。

十步杀一人,千里不留行。

事了拂衣去,深藏身与名。

闲过信陵饮,脱剑膝前横。

将炙啖朱亥,持觞劝侯嬴。

三杯吐然诺,五岳倒为轻。

眼花耳热后,意气素霓生。

救赵挥金锤,邯郸先震惊。

千秋二壮士,烜赫大梁城。

纵死侠骨香,不惭世上英。

谁能书阁下,白首太玄经。
\end{quote}

\hypertarget{PanelData}{%
\chapter{面板数据}\label{PanelData}}

\hypertarget{ux4e00ux53e5ux8bddtips}{%
\section{一句话Tips}\label{ux4e00ux53e5ux8bddtips}}

\begin{itemize}
\tightlist
\item
  \texttt{PSTR}:面板平滑转移模型。
\item
  \texttt{MSCMT}:多个结果变量的合成控制方法的包。
\item
  检查面板数据是否平衡:使用\texttt{table(PanelData{[},1:2{]})}或者\texttt{is.banance}。
\item
  \texttt{phtt}包,交互效应的面板模型,用的Bai (2009)的估计方法。It offers the possibility of analyzing panel data with large dimensions n and T and can be considered when the unobserved heterogeneity effects are time-varying.
\end{itemize}

\hypertarget{plmux5305}{%
\section{\texorpdfstring{\texttt{plm}包}{plm包}}\label{plmux5305}}

\begin{itemize}
\tightlist
\item
  包中的\texttt{vcovG}函数可以计算聚类标准误。一般这么用:
\end{itemize}

\begin{Shaded}
\begin{Highlighting}[]
\KeywordTok{summary}\NormalTok{(plm, }\DataTypeTok{vcov =} \KeywordTok{vcovG}\NormalTok{(plm, }\DataTypeTok{cluster =} \StringTok{'group'}\NormalTok{, }\DataTypeTok{inner =} \StringTok{'cluster'}\NormalTok{))}
\end{Highlighting}
\end{Shaded}

\begin{itemize}
\tightlist
\item
  包中的\texttt{fixef}函数可以返回个体截距项(\texttt{type\ =\ level})。
\item
  \texttt{updata(object,\ formula)}函数可以更新公式重新估计。
\end{itemize}

\hypertarget{ux52a8ux6001ux9762ux677fux9608ux503cux4f30ux8ba1rux8bedux8a00ux4e2dux6709ux4e00ux4e2aux5305dtpux5176ux4f30ux8ba1ux51fdux6570ux4e3a}{%
\section{\texorpdfstring{动态面板阈值估计:R语言中有一个包\texttt{dtp},其估计函数为:}{动态面板阈值估计:R语言中有一个包dtp,其估计函数为:}}\label{ux52a8ux6001ux9762ux677fux9608ux503cux4f30ux8ba1rux8bedux8a00ux4e2dux6709ux4e00ux4e2aux5305dtpux5176ux4f30ux8ba1ux51fdux6570ux4e3a}}

\begin{Shaded}
\begin{Highlighting}[]
\KeywordTok{data}\NormalTok{(Mena)}
\NormalTok{reg<-}\KeywordTok{dtp}\NormalTok{(GDPPC }\OperatorTok{~}\StringTok{ }\NormalTok{FDI}\OperatorTok{+}\NormalTok{OPEN}\OperatorTok{|}\NormalTok{INF}\OperatorTok{|}\NormalTok{INF,Mena,}\DataTypeTok{index=}\KeywordTok{c}\NormalTok{(}\StringTok{"pays"}\NormalTok{,}\StringTok{"ann"}\NormalTok{),}\DecValTok{4}\NormalTok{,}\DecValTok{2}\NormalTok{,}\FloatTok{0.95}\NormalTok{,}\FloatTok{0.8}\NormalTok{,}\DecValTok{1}\NormalTok{,}\DataTypeTok{graph =} \OtherTok{TRUE}\NormalTok{)}
\KeywordTok{summary}\NormalTok{(reg)}
\end{Highlighting}
\end{Shaded}

注意:
- 第一根\texttt{\textbar{}}前的变量是不依赖区制的变量,中间由\texttt{\textbar{}}夹住的变量是阈值变量,最后一个\texttt{\textbar{}}后面的变量是依赖区制的变量(好遗憾,貌似只允许一个这样的变量)。
- \texttt{initnum}参数指的是模型中的内生变量。在动态面板中,一般是因变量的滞后值,因此在数据框中滞后因变量,然后选好该滞后因变量所在列的数字即可。注意,这个数字是在剔除了id和year标识之后的列的序号。
- 数据上千以后,估计过程有点慢,耐心等待。
- 输出中包含一个\texttt{gamma}参数,我揣摩是截距项。

\hypertarget{TS}{%
\chapter{普通回归和时间序列序列}\label{TS}}

\hypertarget{ux4e00ux53e5ux8bddtips-1}{%
\section{一句话Tips}\label{ux4e00ux53e5ux8bddtips-1}}

\begin{itemize}
\tightlist
\item
  \texttt{gdpc}计算广义动态主成分。
\item
  \texttt{POET::POETKhat}提供了计算Bai and NG (2002)因子数目的函数。
\item
  \texttt{confint}函数返回系数的置信区间
\item
  \texttt{bssm}拟合非线性卡尔曼滤波的包。\texttt{pomp},\texttt{KFAS}也是。\texttt{pomp}好像接口更简单些,第四节有一个非线性的例子。
\item
  \texttt{NlinTS}一个利用神经网络的格兰杰因果非线性检验。
\item
  \texttt{slider}: 在任何R数据类型上提供类型稳定的滚动窗口函数,并支持累积窗口和扩展窗口。
\item
  \texttt{testcorr}: 提供计算单变量时间序列中自相关显著性、双变量时间序列中互相关显著性、多变量序列中皮尔逊相关显著性和单变量序列i.i.d.特性的测试统计量的功能。\\
\item
  \texttt{apt}一个阈值协整包。
\item
  \texttt{fDMA}动态平均模型。卡尔曼滤波的贝叶斯模型平均。
\item
  \texttt{MuMIn}利用信息准则进行模型平均的包。
\item
  \texttt{MSBVAR}提供了贝叶斯框架下的马尔科夫转移VAR。\texttt{MSwM}是一个单方程(非单变量)的马尔科夫转移模型估计。
\item
  因子变虚拟变量:\texttt{model.matrix}可以生成回归所需要的矩阵,可以把因子变量变成虚拟变量。
\item
  \texttt{mfGARCH}包估计混频GARCH。
\item
  \texttt{TED::ur.za.fast}和\texttt{urca::ur.za}未知断点的单位根检验。
\item
  \texttt{mFilter}包有各种经济和金融常用的滤波,如HP,BK等滤波(好像没有更新了,官网包的镜像没有找到)。但是可以使用\texttt{FRAPO}包的\texttt{trdhp}函数来计算HP滤波 。
\item
  \texttt{svars}是一个数据驱动的结构VAR包。\texttt{vars}是一个VAR各种估计和诊断的标准包。\texttt{tsDyn}也有线性VAR和VECM的估计,其中它还允许包含外生变量。
\item
  \texttt{lmtest}有\texttt{grangertest()}做双变量格兰杰因果检验。\texttt{MTS::GrangerTest(regdata{[},-c(1,2){]},2,locInput\ =\ 1)}也可以,而且可以做多个变量是不是某个变量的格兰杰原因。\texttt{locInput}表示因变量是第几列。
\item
  \texttt{stats4}包提供了函数\texttt{mle}可以进行极大似然估计,还可以固定部分参数,优化其他参数,这其实是集中似然的思想。关键是它还返回方差协方差矩阵。语法如下,
\end{itemize}

\begin{Shaded}
\begin{Highlighting}[]
\KeywordTok{mle}\NormalTok{(minuslogl, }\DataTypeTok{start =} \KeywordTok{formals}\NormalTok{(minuslogl), }\DataTypeTok{method =} \StringTok{"BFGS"}\NormalTok{,}
    \DataTypeTok{fixed =} \KeywordTok{list}\NormalTok{(), nobs, ...) }\CommentTok{# 注意它的初值是一个list}
\end{Highlighting}
\end{Shaded}

\begin{itemize}
\tightlist
\item
  \texttt{dynlm::dynlm}包一个比\texttt{lm}更强大线性回归结构,优点有三:

  \begin{itemize}
  \tightlist
  \item
    可以使用差分、滞后等表述,如\texttt{d(y)\textasciitilde{}L(y,2)},可以直接添加趋势项\texttt{trend(y)}将使用\texttt{\$(1:n)/Freq\$}作为回归元。
  \item
    可以进行工具变量估计。
    但要注意,他的数据不是数据框,而是一个\texttt{ts}对象。
  \end{itemize}
\item
  \texttt{nardl}估计非线性协整分布滞后模型。
\item
  \texttt{rugarch}:单变量garch建模。一个比\texttt{forcast}更好用的时序建模包。可以用\texttt{show}函数来返回一个丰富的结果,包括一些检验结果。
\item
  \texttt{rmgarch}:多变量garch建模。包括dcc,adcc,gdcc等。
\item
  \texttt{stats}包中的\texttt{ARMAtoMA}函数可以计算AR变成MA。\texttt{vars}包的\texttt{Phi}返回VAR的移动平均系数。
\item
  \texttt{vars}包里面的\texttt{Phi}函数可以把VAR变成VMA。使用\texttt{summary}函数来摘要var的估计结果,会给粗特征根,残差相关矩阵等。
\item
  \texttt{tsDyn}包的\texttt{VECM}函数比较好用,可以包括外生变量,可以选择OLS或Joson方法。这个包也是可以估计线性VAR的,主要是他的\texttt{lineVar}函数。\texttt{egcm}包是恩格尔格兰杰协整检验,这个检验在\texttt{urca}包里业可行。
\item
  \texttt{TSA::periodogram}可以做谱分解。
\item
  \texttt{bvarsv}时变参数var建模
\item
  \texttt{nls}非线性最小二乘法函数
\item
  \texttt{highfrequance}里面有不少意思的函数,包括\texttt{HAR}。
  \#\# 回归中关于公式的理解和构造
\end{itemize}

\begin{Shaded}
\begin{Highlighting}[]
\CommentTok{# 构造公式, 只要包含波浪线就意味着这是一个公式。}
\NormalTok{F1 <-}\StringTok{ }\NormalTok{dist }\OperatorTok{~}\StringTok{ }\NormalTok{speed }\OperatorTok{-}\StringTok{ }\DecValTok{1}
\CommentTok{# 获得公式中所有的变量}
\NormalTok{mf1 <-}\StringTok{ }\KeywordTok{model.frame}\NormalTok{(F1,}\DataTypeTok{data =}\NormalTok{ cars)}
\CommentTok{# 抽取因变量}
\KeywordTok{model.response}\NormalTok{(mf1)}
\CommentTok{# 抽取自变量}
\KeywordTok{model.matrix}\NormalTok{(F1, }\DataTypeTok{data =}\NormalTok{ cars)}
\end{Highlighting}
\end{Shaded}

公式的高级应用还有一个包\texttt{Formula},其说明文件很到位。主要阐述了\texttt{\textbar{}}的使用方式。

\hypertarget{gmmux4f30ux8ba1}{%
\section{GMM估计}\label{gmmux4f30ux8ba1}}

\[
i _t =\beta_0 + \beta_1pi_t + \beta_2GDP_t + \beta_3hp_t + \beta_4i_{t-1} + \varepsilon_t
\]

因包含因变量的滞后项从而有内生性,欲使用\(i_{t-2},i_{t-3},i_{t-4}\)作为工具变量,从而做一个GMM估计,亦即整个方程的矩条件为,

\begin{align}
E(pi_t\varepsilon_t) = 0\\

E(GDP_t\varepsilon_t) = 0\\

E(hp_t\varepsilon_t) = 0\\

E(i_{t-2}\varepsilon_t) = 0\\

E(i_{t-3}\varepsilon_t) = 0\\

E(i_{t-4}\varepsilon_t) = 0
\end{align}

利用这些矩条件的GMM估计在\texttt{gmm}包中的写法为,

\begin{Shaded}
\begin{Highlighting}[]
\NormalTok{gmmrlt <-}\StringTok{ }\KeywordTok{gmm}\NormalTok{(}\DataTypeTok{g =}\NormalTok{ it }\OperatorTok{~}\StringTok{ }\NormalTok{pi }\OperatorTok{+}\StringTok{ }\NormalTok{gdp }\OperatorTok{+}\StringTok{ }\NormalTok{hp }\OperatorTok{+}\StringTok{ }\NormalTok{it1, }\DataTypeTok{x =} \OperatorTok{~}\StringTok{ }\NormalTok{pi }\OperatorTok{+}\StringTok{ }\NormalTok{gdp }\OperatorTok{+}\StringTok{ }\NormalTok{hp }\OperatorTok{+}\StringTok{ }\NormalTok{it2 }\OperatorTok{+}\StringTok{ }\NormalTok{it3 }\OperatorTok{+}\StringTok{ }\NormalTok{it4,}\DataTypeTok{data =}\NormalTok{ dwg0)}
\KeywordTok{summary}\NormalTok{(gmmrlt)}
\end{Highlighting}
\end{Shaded}

其中\texttt{g}可直接写成公式,\texttt{x}即为工具变量集。

\hypertarget{ux5b63ux8282ux8c03ux6574}{%
\section{季节调整}\label{ux5b63ux8282ux8c03ux6574}}

\hypertarget{rux4e2dux6709x12ux5305ux53efux4ee5ux505aux5b63ux8282ux5904ux7406}{%
\subsection{\texorpdfstring{R中有\texttt{x12}包可以做季节处理}{R中有x12包可以做季节处理}}\label{rux4e2dux6709x12ux5305ux53efux4ee5ux505aux5b63ux8282ux5904ux7406}}

注意:

\begin{itemize}
\tightlist
\item
  要先下载美国统计局的x12程序包,并在调用函数时,记得写上所以存储的路径。
\item
  仅可处理R中内置的时间序列对象\texttt{ts}。
\item
  示例代码:
\end{itemize}

\begin{Shaded}
\begin{Highlighting}[]
\KeywordTok{library}\NormalTok{(x12)}
\KeywordTok{data}\NormalTok{(AirPassengers)}
\NormalTok{x12out <-}\StringTok{ }\KeywordTok{x12work}\NormalTok{(AirPassengers,}
             \DataTypeTok{x12path =} \StringTok{'C:}\CharTok{\textbackslash{}\textbackslash{}}\StringTok{ado}\CharTok{\textbackslash{}\textbackslash{}}\StringTok{plus}\CharTok{\textbackslash{}\textbackslash{}}\StringTok{WinX12}\CharTok{\textbackslash{}\textbackslash{}}\StringTok{x12a}\CharTok{\textbackslash{}\textbackslash{}}\StringTok{x12a.exe'}\NormalTok{,}\DataTypeTok{keep_x12out =} \OtherTok{FALSE}\NormalTok{)}
\NormalTok{x12out}\OperatorTok{$}\NormalTok{d11 }\CommentTok{#此即为调整后的时间序列}
\end{Highlighting}
\end{Shaded}

其中,\texttt{keep\_x12out}参数表示是否要保留计算后的文件。

\hypertarget{seasonalux5305ux6709x13ux5904ux7406ux66f4ux52a0ux4fbfux6377}{%
\subsection{\texorpdfstring{\texttt{seasonal}包有x13处理,更加便捷}{seasonal包有x13处理,更加便捷}}\label{seasonalux5305ux6709x13ux5904ux7406ux66f4ux52a0ux4fbfux6377}}

\begin{Shaded}
\begin{Highlighting}[]
\KeywordTok{library}\NormalTok{(seasonal)}
\NormalTok{m <-}\StringTok{ }\KeywordTok{seas}\NormalTok{(AirPassengers) }\CommentTok{# x13 处理, AirPassengers是一个ts对象}
\KeywordTok{final}\NormalTok{(m) }\CommentTok{# 最终调整序列}
\KeywordTok{plot}\NormalTok{(m) }\CommentTok{# 绘制调整和未调整序列}
\end{Highlighting}
\end{Shaded}

\hypertarget{DataProcess}{%
\chapter{数据处理}\label{DataProcess}}

\hypertarget{ux4e00ux53e5ux8bddtips-2}{%
\section{一句话Tips}\label{ux4e00ux53e5ux8bddtips-2}}

\begin{itemize}
\tightlist
\item
  因子操作
\end{itemize}

\begin{Shaded}
\begin{Highlighting}[]
\CommentTok{# 使用字符串有两个缺陷:第一,不在因子水平范围内的不会转化成NA}
\CommentTok{# 第二,仅按字母排序。}
\CommentTok{# 因此,通过设定因子水平,可以解决上述两个问题。注意水平和字符串是一样的,}
\CommentTok{# 只是相当于设定了范围和排序。}
    \KeywordTok{factor}\NormalTok{(}\KeywordTok{c}\NormalTok{(}\StringTok{'Dec'}\NormalTok{,}\StringTok{'Apr'}\NormalTok{,}\StringTok{'Jam'}\NormalTok{,}\StringTok{'Mar'}\NormalTok{), }\DataTypeTok{levels =}\NormalTok{ (}\StringTok{'Jan'}\NormalTok{,}\StringTok{'Feb'}\NormalTok{,}\StringTok{'Mar'}\NormalTok{,}\StringTok{'Apr'}\NormalTok{,}\StringTok{'May'}\NormalTok{))}
\CommentTok{# 因子重编码, 把1改成unmarried等}
\NormalTok{farcats}\OperatorTok{::}\KeywordTok{fct_recode}\NormalTok{(rawdata}\OperatorTok{$}\NormalTok{marrige,}
           \StringTok{'unmarried'}\NormalTok{=}\StringTok{'1'}\NormalTok{,}\StringTok{'married'}\NormalTok{=}\StringTok{'2'}\NormalTok{,}\StringTok{'cohabitation'}\NormalTok{=}\StringTok{'3'}\NormalTok{,}\StringTok{'divore'}\NormalTok{=}\StringTok{'4'}\NormalTok{,}\StringTok{'wid'}\NormalTok{=}\StringTok{'5'}\NormalTok{)}
\end{Highlighting}
\end{Shaded}

\begin{itemize}
\tightlist
\item
  \texttt{dbplyr}可以连接到几乎任何数据库。
\item
  \texttt{wbstats}下载世界银行数据,很牛逼。Stata里面的\texttt{wbopendata}包更牛逼。
\item
  \texttt{stationaRy}:一个从NOAA上下载气象数据,如气温,风向等的包。该包就三个函数,一个用来得到站点id,一个用这个id下载数据,还有一个是如果你想得到其他额外的气象数据时可能有用。
\item
  \textbf{当你发现你用\texttt{save}命令保存一个数据长达数分钟时},建议你迅速调用\texttt{qs}包,可能一分钟不到就帮你快速读入和保存了。但这个包一次只能保存一个变量。
\item
  \texttt{tor}: 提供允许用户同时导入多个文件的功能.
\item
  读入excel中的sheet名:\texttt{openxlsx::getSheetNames(file)}
\item
  \texttt{XLConect}处理excel最强大的包。但需要JRE(java run enviornment)。
\end{itemize}

\begin{Shaded}
\begin{Highlighting}[]
\CommentTok{# 可以不改变原有数据,然后把一个数据框精准地写入某个地方}
\KeywordTok{writeWorksheetToFile}\NormalTok{(}\StringTok{"XLConnectExample2.xlsx"}\NormalTok{, }\DataTypeTok{data =}\NormalTok{ ChickWeight,}
 \DataTypeTok{sheet =} \StringTok{"chickSheet"}\NormalTok{, }\DataTypeTok{startRow =} \DecValTok{3}\NormalTok{, }\DataTypeTok{startCol =} \DecValTok{4}\NormalTok{,}\DataTypeTok{header =} \OtherTok{FALSE}\NormalTok{, }\DataTypeTok{clearSheets =} \OtherTok{FALSE}\NormalTok{)}
\end{Highlighting}
\end{Shaded}

\begin{itemize}
\item
  使用\texttt{as.Date}来生成日期,必须带有年月日三个要素,使用\texttt{format}来输出日期格式,此时可以只输出年和月。如\texttt{as.Date(\textquotesingle{}2010/05/01\textquotesingle{})\ \%\textgreater{}\%\ format(.,format\ =\ \textquotesingle{}\%Y\%m\textquotesingle{})}
\item
  \texttt{seq.Date()}生成日期序列,包括日、星期、月、年。
\item
  \texttt{readstata13}包可以读入更高版本的stata数据格式。
\item
  \texttt{zoo::rollapply(x,\ 30,\ mean)}就是30天的移动平均求值。
\item
  \texttt{select}是一个很牛逼的函数
\end{itemize}

\begin{Shaded}
\begin{Highlighting}[]
\KeywordTok{select}\NormalTok{(regdata,id, year) }\CommentTok{# 选择regdata数据框的id和year两列}
\KeywordTok{select}\NormalTok{(regdata,}\KeywordTok{starts_with}\NormalTok{(}\StringTok{'abc'}\NormalTok{)) }\CommentTok{# 匹配以'abc'开头的列}
\KeywordTok{select}\NormalTok{(regdata,}\KeywordTok{ends_with}\NormalTok{(}\StringTok{'abc'}\NormalTok{)) }\CommentTok{# 匹配以'abc'结尾的列}
\KeywordTok{select}\NormalTok{(regdata,}\KeywordTok{contains}\NormalTok{(}\StringTok{'abc'}\NormalTok{)) }\CommentTok{# 匹配包含'abc'的列}
\KeywordTok{select}\NormalTok{(regdata,}\KeywordTok{matches}\NormalTok{(}\StringTok{'abc'}\NormalTok{)) }\CommentTok{# 正则表达匹配}
\KeywordTok{select}\NormalTok{(regdata,}\KeywordTok{num_range}\NormalTok{(}\StringTok{'x'}\NormalTok{,}\DecValTok{1}\OperatorTok{:}\DecValTok{3}\NormalTok{)) }\CommentTok{# 匹配x1, x2,x3的列}
\end{Highlighting}
\end{Shaded}

\begin{itemize}
\tightlist
\item
  R语言给数组各维数命名
\end{itemize}

\begin{Shaded}
\begin{Highlighting}[]
\CommentTok{# Create two vectors of different lengths.}
\NormalTok{vector1 <-}\StringTok{ }\KeywordTok{c}\NormalTok{(}\DecValTok{5}\NormalTok{,}\DecValTok{9}\NormalTok{,}\DecValTok{3}\NormalTok{)}
\NormalTok{vector2 <-}\StringTok{ }\KeywordTok{c}\NormalTok{(}\DecValTok{10}\NormalTok{,}\DecValTok{11}\NormalTok{,}\DecValTok{12}\NormalTok{,}\DecValTok{13}\NormalTok{,}\DecValTok{14}\NormalTok{,}\DecValTok{15}\NormalTok{)}
\NormalTok{column.names <-}\StringTok{ }\KeywordTok{c}\NormalTok{(}\StringTok{"COL1"}\NormalTok{,}\StringTok{"COL2"}\NormalTok{,}\StringTok{"COL3"}\NormalTok{)}
\NormalTok{row.names <-}\StringTok{ }\KeywordTok{c}\NormalTok{(}\StringTok{"ROW1"}\NormalTok{,}\StringTok{"ROW2"}\NormalTok{,}\StringTok{"ROW3"}\NormalTok{)}
\NormalTok{matrix.names <-}\StringTok{ }\KeywordTok{c}\NormalTok{(}\StringTok{"Matrix1"}\NormalTok{,}\StringTok{"Matrix2"}\NormalTok{)}

\CommentTok{# Take these vectors as input to the array.}
\NormalTok{result <-}\StringTok{ }\KeywordTok{array}\NormalTok{(}\KeywordTok{c}\NormalTok{(vector1,vector2),}\DataTypeTok{dim =} \KeywordTok{c}\NormalTok{(}\DecValTok{3}\NormalTok{,}\DecValTok{3}\NormalTok{,}\DecValTok{2}\NormalTok{),}\DataTypeTok{dimnames =} \KeywordTok{list}\NormalTok{(row.names,column.names,}
\NormalTok{                                                                  matrix.names))}
\KeywordTok{print}\NormalTok{(result)}
\end{Highlighting}
\end{Shaded}

\begin{itemize}
\tightlist
\item
  \texttt{pdftools}包的函数可以读PDF文件:
\end{itemize}

\begin{Shaded}
\begin{Highlighting}[]
\KeywordTok{pdf_info}\NormalTok{(pdf, }\DataTypeTok{opw =} \StringTok{""}\NormalTok{, }\DataTypeTok{upw =} \StringTok{""}\NormalTok{)}

\KeywordTok{pdf_text}\NormalTok{(pdf, }\DataTypeTok{opw =} \StringTok{""}\NormalTok{, }\DataTypeTok{upw =} \StringTok{""}\NormalTok{)}

\KeywordTok{pdf_data}\NormalTok{(pdf, }\DataTypeTok{opw =} \StringTok{""}\NormalTok{, }\DataTypeTok{upw =} \StringTok{""}\NormalTok{)}

\KeywordTok{pdf_fonts}\NormalTok{(pdf, }\DataTypeTok{opw =} \StringTok{""}\NormalTok{, }\DataTypeTok{upw =} \StringTok{""}\NormalTok{)}

\KeywordTok{pdf_attachments}\NormalTok{(pdf, }\DataTypeTok{opw =} \StringTok{""}\NormalTok{, }\DataTypeTok{upw =} \StringTok{""}\NormalTok{)}

\KeywordTok{pdf_toc}\NormalTok{(pdf, }\DataTypeTok{opw =} \StringTok{""}\NormalTok{, }\DataTypeTok{upw =} \StringTok{""}\NormalTok{)}

\KeywordTok{pdf_pagesize}\NormalTok{(pdf, }\DataTypeTok{opw =} \StringTok{""}\NormalTok{, }\DataTypeTok{upw =} \StringTok{""}\NormalTok{)}
\end{Highlighting}
\end{Shaded}

同时,利用\texttt{qpdf}包的\texttt{pdf\_subset,pdf\_combine,pdf\_split}可以提取PDF的部分内容,合并PDF文件,把每一页分成一个PDF文件。

\hypertarget{rjsdmxux5305ux4e0bux8f7dux4e16ux754cux5404ux5927ux6570ux636eux5e93ux6570ux636e}{%
\section{\texorpdfstring{\texttt{RJSDMX}包下载世界各大数据库数据}{RJSDMX包下载世界各大数据库数据}}\label{rjsdmxux5305ux4e0bux8f7dux4e16ux754cux5404ux5927ux6570ux636eux5e93ux6570ux636e}}

一般工作流:

\begin{Shaded}
\begin{Highlighting}[]
\KeywordTok{library}\NormalTok{(RJSDMX)}
\CommentTok{# 查看有哪些库可以用}
\KeywordTok{getProviders}\NormalTok{()}
\CommentTok{# 库中有哪些子库可以用}
\KeywordTok{getFlows}\NormalTok{(}\StringTok{'WITS'}\NormalTok{)}
\CommentTok{# 该子库调取数据需要哪几个字段}
\KeywordTok{getDimensions}\NormalTok{(}\StringTok{'WITS'}\NormalTok{,}\StringTok{'WBG_WITS,DF_WITS_TradeStats_Tariff,1.0'}\NormalTok{)}
\CommentTok{# 查看这个指标有几个选项 }
\KeywordTok{getCodes}\NormalTok{(}\StringTok{'WITS'}\NormalTok{,}\StringTok{'WBG_WITS,DF_WITS_TradeStats_Tariff,1.0'}\NormalTok{,}\StringTok{'INDICATOR'}\NormalTok{)}
\CommentTok{# 查好了就可以下载}
\NormalTok{ans <-}\StringTok{ }\KeywordTok{getTimeSeries}\NormalTok{(}\StringTok{'WITS'}\NormalTok{, }\StringTok{'DF_WITS_TradeStats_Tariff/A.CHN.WLD.01-05_Animal.MFN-WGHTD-AVRG'}\NormalTok{)}
\CommentTok{# 你也可以调用图形窗口查阅命令}
\KeywordTok{sdmxHelp}\NormalTok{()}
\end{Highlighting}
\end{Shaded}

\texttt{IMF2}里面的\texttt{IFS}数据库里面有很多季度的宏观数据,如GDP,固定资本形成等

\hypertarget{ux7edfux8ba1}{%
\chapter{统计}\label{ux7edfux8ba1}}

\hypertarget{ux4e00ux53e5ux8bddtips-3}{%
\section{一句话Tips}\label{ux4e00ux53e5ux8bddtips-3}}

\begin{itemize}
\tightlist
\item
  \texttt{cmna::mcint}可以进行蒙特卡洛积分。
\item
  数值积分:\texttt{pracma::integral}
\item
  多元正态分布随机抽样:\texttt{SimDesign::rmvnorm},还有\texttt{mvnfast}
\item
  \texttt{KSgeneral}包执行KS检验,比较一个分布是否来自某个理论分布。\texttt{stats}包的\texttt{ks.test}和\texttt{dgof}包的\texttt{ks.test}也可以,并且可以比较双样本是否来自同一个分布。
\item
  \texttt{choose(n,k)}:组合公式,n个里面选k个,有多少种组合方式。\texttt{utils::combn(n,k)}也可以。 \texttt{e1071::permutations}实现排列。
\item
  \texttt{qrandom}: 利用量子波动产生真随机数.
\item
  主成分分析可以调用\texttt{psych}包两个步骤实现:
\end{itemize}

\begin{Shaded}
\begin{Highlighting}[]
\CommentTok{# 画个图选特征值数目:}
\CommentTok{# 1. 特征值在1以上的才行; 2. 特征值大于模拟的平均特征才可行; 3. 碎石图}
\KeywordTok{library}\NormalTok{(psych)}
\KeywordTok{fa.parallel}\NormalTok{(regdata, }\DataTypeTok{fa =} \StringTok{'pc'}\NormalTok{)}
\CommentTok{# 计算2个主成分。如果想要主成分载荷更有经济意义,注意设置旋转参数}
 \KeywordTok{principal}\NormalTok{(regdata,}\DataTypeTok{nfactors =} \DecValTok{2}\NormalTok{,}\DataTypeTok{rotate =} \StringTok{'none'}\NormalTok{)}
\end{Highlighting}
\end{Shaded}

\hypertarget{mcmcux7b97ux6cd5}{%
\section{MCMC算法}\label{mcmcux7b97ux6cd5}}

\hypertarget{ux5409ux5e03ux65afux62bdux6837ux539fux7406}{%
\subsection{吉布斯抽样原理}\label{ux5409ux5e03ux65afux62bdux6837ux539fux7406}}

如果联合分布不好求,但条件分布好求,可以用这个算法。

\hypertarget{ux4e00ux4e9bux5171ux8f6dux5148ux9a8cux5206ux5e03ux7684ux7ed3ux8bba}{%
\subsection{一些共轭先验分布的结论}\label{ux4e00ux4e9bux5171ux8f6dux5148ux9a8cux5206ux5e03ux7684ux7ed3ux8bba}}

理解这些结论,对于后续使用吉布斯抽样、MH算法非常有用。

\textbf{结论1} 若\(x_1,\cdots,x_n\)是从均值为\(\mu\)(\textbf{未知}),方差为\(\sigma^2\)(\textbf{已知}且为正)中正态分布中所抽取的一个随机样本,同时假定\(\mu\sim \mathcal{N}(\mu_0,\sigma_0^2)\),则给定数据和先验分布,\(\mu\)的后验分布也是一个正态分布,其后验均值和方差为,
\[\mu_* = \frac{\sigma^2\mu_0+n\sigma_0^2\overline x}{\sigma^2+n\sigma_0^2},\hspace{2em}\sigma_*=\frac{\sigma^2\sigma^2_0}{\sigma^2+n\sigma^2_0},\;\;\;\text{其中},\overline x= \sum_i^n x_i/n\]

推广到多变量,则可以写为,
\[{\mu}_*=\Sigma_*(\Sigma_0^{-1}{\mu}_0+\Sigma^{-1}\overline{\bf{x}}), \hspace{2em}\Sigma_*^{-1} = \Sigma_0^{-1}+n\Sigma^{-1}\]

\textbf{结论2} 若\(e_1,\cdots,e_n\)是从均值为0,方差为\(\sigma^2\)的正态分布中抽取的随机样本,同时假定\(\sigma^2\)的先验分布是自由度为\(\nu\)的逆\(\chi^2\)分布,即\(\frac{\nu\lambda}{\sigma^2}\sim \chi^2_\nu,\lambda>0\),则\(\sigma^2\)的后验分布也是逆\(\chi^2\)分布,自由度为\(\nu+n\),
\[\frac{\nu\lambda+\sum_i^ne_i^2}{\sigma^2}\sim \chi^2_{\nu+n}\]

\hypertarget{ux4e00ux4e2aux5409ux5e03ux65afux62bdux6837ux7684ux5178ux578bux6848ux4f8b}{%
\subsection{一个吉布斯抽样的典型案例}\label{ux4e00ux4e2aux5409ux5e03ux65afux62bdux6837ux7684ux5178ux578bux6848ux4f8b}}

一个带自相关的回归模型可以写为,
\begin{align}
y_t&=\beta_0+\beta_1x_{1t}+\cdots+\beta_kx_{kt}+z_t\\
z_t&=\phi z_{t-1}+e_t
\end{align}

该模型需要估计的参数有三个,即\(\theta = (\beta',\phi,\sigma^2)\)。该参数的联合分布并不好求,但是条件分布则好求得多。

\hypertarget{metropolis-ux548c-m-hux7b97ux6cd5}{%
\subsection{Metropolis 和 M-H算法}\label{metropolis-ux548c-m-hux7b97ux6cd5}}

如果后验分布除了那个归一化的常数不知道,但分子是知道的,那可以用这个算法。这个场景是不是在贝叶斯估计中很熟悉?

\texttt{MCMCpack::MCMCmetrop1R}中有个例子提供了Metropolis算法,感觉还是很清晰。里面提到的'The proposal distribution'其实就是跳跃分布,即给定上一次抽样的参数,从这个跳跃分布中抽下一个参数。

\hypertarget{ux4e00ux4e9bux5e26ux8d1dux53f6ux65afux4f30ux8ba1ux7684rux5305ux4f7fux7528ux62a5ux544a}{%
\subsection{一些带贝叶斯估计的R包使用报告}\label{ux4e00ux4e9bux5e26ux8d1dux53f6ux65afux4f30ux8ba1ux7684rux5305ux4f7fux7528ux62a5ux544a}}

\begin{itemize}
\tightlist
\item
  \texttt{MTS::BVAR}:这个包可以在一个一般的先验设定上估计VAR,先验可以是乏信息先验,也可以是明尼苏达先验,但问题是该包仅返回估计系数的均值和标准误,不返回抽样。
\item
  \texttt{bvartools}:在很大程度上可以定制BVAR的mcmc抽样,见它的一个优秀的引言。我用这个,自己写了乏信息先验的BVAR估计包。下次我再把明尼苏达先验添进去。
\item
  \texttt{MCMCpack::MCMCregress}:单方程的贝叶斯估计,它提供了\(\beta\)是多元正态先验,方程误差项的方差协方差是逆伽玛的先验估计。
\item
  \texttt{bayesm::runireg}:单方程的贝叶斯估计,它提供了\(\beta\)是多元正态先验,方程误差项的方差协方差是卡方分布的先验估计。
\end{itemize}

\hypertarget{rawR}{%
\chapter{原生的R}\label{rawR}}

\hypertarget{ux4e00ux53e5ux8bddtips-4}{%
\section{一句话Tips}\label{ux4e00ux53e5ux8bddtips-4}}

\begin{itemize}
\tightlist
\item
  `\texttt{+}(3,1)\texttt{与}3+1\texttt{的作用是一样的。这可以推广到其他中缀函数,如}\%*\%`等。
\item
  写函数时,可以使用\texttt{...}参数,为捕获这个(可能是多个)参数的值,可以用\texttt{list(...)}这个办法。
\item
  只显示3位小数:
\end{itemize}

\begin{Shaded}
\begin{Highlighting}[]
\KeywordTok{round}\NormalTok{(}\FloatTok{0.123456}\NormalTok{,}\DecValTok{3}\NormalTok{)}
\end{Highlighting}
\end{Shaded}

\begin{verbatim}
## [1] 0.123
\end{verbatim}

\begin{itemize}
\tightlist
\item
  属性赋值:
\end{itemize}

\begin{Shaded}
\begin{Highlighting}[]
\NormalTok{y <-}\StringTok{ }\KeywordTok{c}\NormalTok{(}\DecValTok{1}\NormalTok{,}\DecValTok{2}\NormalTok{,}\DecValTok{5}\NormalTok{,}\DecValTok{8}\NormalTok{)}
\KeywordTok{attr}\NormalTok{(y,}\StringTok{'my_attribute'}\NormalTok{) <-}\StringTok{ 'This is a vector'}
\KeywordTok{attr}\NormalTok{(y,}\StringTok{'my_attribute'}\NormalTok{)}
\end{Highlighting}
\end{Shaded}

\begin{verbatim}
## [1] "This is a vector"
\end{verbatim}

\begin{itemize}
\tightlist
\item
  \texttt{remove.packages(\textquotesingle{}dplyr\textquotesingle{})},卸载已安装的包。
\item
  \texttt{system}或\texttt{shell}运行Shell命令。
\item
  更新所有的包\texttt{update.packages(checkBuilt=TRUE,\ ask=FALSE)}
\item
  \texttt{pkgsearch}包的\texttt{ps}函数提供CRAN的关键词搜寻。
\item
  \texttt{detach(package:dplyr)}可以去掉加载的包。
\item
  \texttt{foreach}包提供循环的平行计算
\item
  在jupyter里面安装R,只需在anaconda里面的命令行中(anaconda prompt)输入,
\end{itemize}

\begin{verbatim}
conda install -c r r-essentials
\end{verbatim}

\begin{itemize}
\tightlist
\item
  工作目录下所有文件名\texttt{dir()}
\item
  \texttt{file.copy,\ file.create,\ file.remove,\ file.rename,\ dir.creat,\ file.exists,\ file.info}
\item
  \texttt{file.rename}批量修改文件名
\end{itemize}

\begin{Shaded}
\begin{Highlighting}[]
\NormalTok{fr =}\StringTok{ }\KeywordTok{paste}\NormalTok{(}\StringTok{'./加工贸易HS/2016/'}\NormalTok{,}\KeywordTok{dir}\NormalTok{(}\StringTok{'./加工贸易HS/2016'}\NormalTok{),}\DataTypeTok{sep =} \StringTok{''}\NormalTok{)}
\NormalTok{to =}\StringTok{ }\KeywordTok{paste}\NormalTok{(}\StringTok{'./加工贸易HS/2016/hp'}\NormalTok{,}\KeywordTok{dir}\NormalTok{(}\StringTok{'./加工贸易HS/2016'}\NormalTok{),}\DataTypeTok{sep =} \StringTok{''}\NormalTok{)}
\KeywordTok{file.rename}\NormalTok{(}\DataTypeTok{from =}\NormalTok{ fr,}\DataTypeTok{to =}\NormalTok{ to)}
\end{Highlighting}
\end{Shaded}

\begin{itemize}
\tightlist
\item
  \texttt{down.file}只要给出第一个参数:网址(包括ftp的)和第二个参数,下载的文件要保存的文件名,就可以直接在网上下载文件。如果中国乱码,记得使用fileEncoding = 'UTF-8'来修正。
\item
  \texttt{getAnywhere(predict.Arima)}查看源代码
\item
  \texttt{.rs.restartR()}重启一个新的R会话
\item
  平行计算。光使用\texttt{foreach}包是不够的,还需要注册一个平行背景注册,否则\texttt{foreach}包在运算完以后会返回警告:
\end{itemize}

\begin{quote}
Warning message:

executing \%dopar\% sequentially: no parallel backend registered
\end{quote}

A: 如何注册呢?调用\texttt{doParallel}包,代码如下:

\begin{Shaded}
\begin{Highlighting}[]
\KeywordTok{library}\NormalTok{(doParallel)}
\NormalTok{cl <-}\StringTok{ }\KeywordTok{makeCluster}\NormalTok{(}\DecValTok{2}\NormalTok{)}
\KeywordTok{registerDoParallel}\NormalTok{(cl)}
\KeywordTok{foreach}\NormalTok{(}\DataTypeTok{i=}\DecValTok{1}\OperatorTok{:}\DecValTok{3}\NormalTok{, }\DataTypeTok{.pacakages =} \StringTok{'tidyverse'}\NormalTok{) }\OperatorTok\StringTok{ }\KeywordTok{sqrt}\NormalTok{(i)}
\KeywordTok{stopCluster}\NormalTok{(cl)}
\end{Highlighting}
\end{Shaded}

\begin{itemize}
\tightlist
\item
  如何安装已经过期的包?

  \begin{enumerate}
  \def\labelenumi{\arabic{enumi}.}
  \tightlist
  \item
    点\href{https://cran.r-project.org/src/contrib/Archive/}{这里}找到过期的包,然后下载下来。
  \item
    用这个命令安装本地的包:\texttt{install.packages(\textquotesingle{}D:/MSBVAR\_0.9-3.tar.gz\textquotesingle{},repos\ =\ NULL,\ type\ =\ \textquotesingle{}source\textquotesingle{})}
  \end{enumerate}
\end{itemize}

\hypertarget{ux7c7bux548cux65b9ux6cd5}{%
\section{类和方法}\label{ux7c7bux548cux65b9ux6cd5}}

\hypertarget{s3ux7c7b}{%
\subsection{S3类}\label{s3ux7c7b}}

\begin{Shaded}
\begin{Highlighting}[]
\CommentTok{# 查看属于一个泛型函数的所有方法:}
\KeywordTok{methods}\NormalTok{(}\StringTok{'mean'}\NormalTok{)}
\end{Highlighting}
\end{Shaded}

\begin{verbatim}
## [1] mean.Date        mean.default     mean.difftime    mean.POSIXct    
## [5] mean.POSIXlt     mean.quosure*    mean.vctrs_vctr*
## see '?methods' for accessing help and source code
\end{verbatim}

\begin{Shaded}
\begin{Highlighting}[]
\CommentTok{# 反过来,查看一个类,都有何关联的泛型函数}
\KeywordTok{methods}\NormalTok{(}\DataTypeTok{class =} \StringTok{'ts'}\NormalTok{)}
\end{Highlighting}
\end{Shaded}

\begin{verbatim}
##  [1] [             [<-           aggregate     as.data.frame as_tibble    
##  [6] cbind         coerce        cycle         diff          diffinv      
## [11] filter        initialize    kernapply     lines         Math         
## [16] Math2         monthplot     na.omit       Ops           plot         
## [21] print         show          slotsFromS3   t             time         
## [26] window        window<-     
## see '?methods' for accessing help and source code
\end{verbatim}

创建一个类,很多时候只需在最后返回一个这样的,就可以了,

\begin{Shaded}
\begin{Highlighting}[]
\KeywordTok{class}\NormalTok{(foo) <-}\StringTok{ 'myclass'}   
\end{Highlighting}
\end{Shaded}

然后为这个类创建一个泛型函数,只需要两步:

\begin{Shaded}
\begin{Highlighting}[]
\CommentTok{# 创建一个类}
\NormalTok{a <-}\StringTok{ }\KeywordTok{list}\NormalTok{()}
\KeywordTok{class}\NormalTok{(a) <-}\StringTok{ 'a'}
\CommentTok{# 第一步:增加一个新的泛型函数。记住,没有搭配该泛型函数的方法,泛型函数是没有用的。}
\NormalTok{f <-}\StringTok{ }\ControlFlowTok{function}\NormalTok{(x) }\KeywordTok{UseMethod}\NormalTok{(}\StringTok{'f'}\NormalTok{) }
\CommentTok{# 第二步,为此泛型函数添加方法。关键在于命名规则,属于该泛型函数的方法一定具有类似f.a格式的命名。}
\CommentTok{# f是泛型函数, a是类,它们用点连起来。}
\NormalTok{f.a <-}\StringTok{ }\ControlFlowTok{function}\NormalTok{(x) }\StringTok{'class a'}
\NormalTok{mean.a <-}\StringTok{ }\ControlFlowTok{function}\NormalTok{(x) }\StringTok{'a'} \CommentTok{# 为已有的泛型函数增加方法}
\end{Highlighting}
\end{Shaded}

\hypertarget{ux6253ux5370ux5230ux6587ux4ef6}{%
\section{打印到文件}\label{ux6253ux5370ux5230ux6587ux4ef6}}

\begin{itemize}
\tightlist
\item
  \texttt{sink}函数:在代码开始前加一行:\texttt{sink(“output.txt”)},就会自动把结果全部输出到工作文件夹下的output.txt文本文档。这时在R控制台的输出窗口中是看不到输出结果的。代码结束时用\texttt{sink()}切换回来。 示例:
\end{itemize}

\begin{Shaded}
\begin{Highlighting}[]
\KeywordTok{sink}\NormalTok{(}\StringTok{"a.txt"}\NormalTok{) }
\NormalTok{x<-}\KeywordTok{rnorm}\NormalTok{(}\DecValTok{100}\NormalTok{,}\DecValTok{0}\NormalTok{,}\DecValTok{1}\NormalTok{) }
\KeywordTok{mean}\NormalTok{(x) }
\KeywordTok{sink}\NormalTok{()}
\end{Highlighting}
\end{Shaded}

\begin{itemize}
\tightlist
\item
  \texttt{cat}函数:\texttt{cat(\textquotesingle{}abc\textquotesingle{},\textquotesingle{}OK!\textquotesingle{},file\ =\ \textquotesingle{}a.txt\textquotesingle{},sep\ =\ \textquotesingle{}\textbackslash{}n\textquotesingle{},append\ =\ T)}
\item
  \texttt{stargazer}函数:
\end{itemize}

\begin{Shaded}
\begin{Highlighting}[]
\KeywordTok{stargazer}\NormalTok{(fit1, fit2, }\DataTypeTok{title =} \StringTok{"results"}\NormalTok{, }\DataTypeTok{align =}\NormalTok{ F, }\DataTypeTok{type =} \StringTok{"text"}\NormalTok{, }\DataTypeTok{no.space =} \OtherTok{TRUE}\NormalTok{, }\DataTypeTok{out =} \StringTok{"fit.html"}\NormalTok{)}
\end{Highlighting}
\end{Shaded}

\hypertarget{Eco}{%
\chapter{经济学中的各种专业计算}\label{Eco}}

\hypertarget{ux4e00ux53e5ux8bddtips-5}{%
\section{一句话Tips}\label{ux4e00ux53e5ux8bddtips-5}}

\begin{itemize}
\tightlist
\item
  \texttt{lpirfs}包局部线性投影脉冲响应函数。
\item
  \texttt{library(productivity)}计算满奎斯特效率指数,注意日期参数\texttt{time.var}以及个体\texttt{id.var}参数都要是整数。
\end{itemize}

\begin{Shaded}
\begin{Highlighting}[]
\NormalTok{tfp <-}\StringTok{ }\KeywordTok{malm}\NormalTok{(regdata, }\DataTypeTok{id.var =} \StringTok{'alphabets'}\NormalTok{,}\DataTypeTok{time.var =} \StringTok{'yr'}\NormalTok{, }\DataTypeTok{x.vars =} \KeywordTok{c}\NormalTok{(}\StringTok{'wage'}\NormalTok{,}\StringTok{'K'}\NormalTok{), }\DataTypeTok{y.vars =} \StringTok{'gdp'}\NormalTok{)}
\KeywordTok{Changes}\NormalTok{(tfp) }\CommentTok{# 获得malmquist指数及其成份}
\end{Highlighting}
\end{Shaded}

\begin{itemize}
\tightlist
\item
  产品编码之间的转换包:\texttt{concordance::concord}:
  It supports concordance between HS (Combined), ISIC Rev.~2,3, and SITC1,2,3,4 product classification codes, as well as BEC, NAICS, and SIC classifications. It also provides code nomenclature / descriptions look-up, Rauch classification look-up (via concordance to SITC2) and trade elasticity look-up (via concordance to SITC2/3 or HS3.ss).
\item
  \texttt{ioanalysis}提供投入产出表的分析功能
\item
  出口增加值分解的包由\texttt{decompr},这个包有Wang et at. (2013)的分解以及经典的里昂惕夫分解。以及\texttt{gvc}包。
\item
  \texttt{hhi}包可以算赫芬达尔指数。
\end{itemize}

\hypertarget{wwzux7684ux8d38ux6613ux589eux52a0ux503cux5206ux89e3}{%
\section{wwz的贸易增加值分解}\label{wwzux7684ux8d38ux6613ux589eux52a0ux503cux5206ux89e3}}

分解的主要函数是\texttt{decomp}。但是有一个\texttt{load\_table\_vectors}函数可以生成一个decompr class,这个类有很多我们想要的东西,如投入产出系数A,里昂惕夫矩阵B以及其他的一些数据。

\begin{Shaded}
\begin{Highlighting}[]
\CommentTok{# load example data}
\KeywordTok{data}\NormalTok{(leather)}

\CommentTok{# create intermediate object (class decompr)}
\NormalTok{decompr_object <-}\StringTok{ }\KeywordTok{load_tables_vectors}\NormalTok{(inter,}
\NormalTok{                                      final,}
\NormalTok{                                      countries,}
\NormalTok{                                      industries,}
\NormalTok{                                      out        )}
\end{Highlighting}
\end{Shaded}

这个类包含了如下内容,我挑一些我确切知道是啥的:

\begin{itemize}
\tightlist
\item
  B: 里昂惕夫矩阵,即\((I-A)^{-1}\)
\item
  Vc: 增加值系数矩阵
\item
  ESR: 总出口,包括中间品出口\texttt{Eint}和最终品\texttt{Efd}出口.
\item
  L: 单个国家的里昂惕夫逆矩阵
\item
  Y: 最终需求。几个国家就几列。
\item
  Yd: 自己对自己的需求。也就是在这个矩阵里面,自己对别人的需求都是0。
\item
  Ym: 自己对别人的需求。自己对自己的需求都是0。
\item
  Bm, Bd: 含义比照Ym, Yd.
\end{itemize}

\hypertarget{math}{%
\chapter{数学计算}\label{math}}

\hypertarget{ux4e00ux53e5ux8bddtips-6}{%
\section{一句话Tips}\label{ux4e00ux53e5ux8bddtips-6}}

\begin{itemize}
\tightlist
\item
  \texttt{crossprod(x,y)}的意思是\(x'y\),\texttt{tcrossprod(x,y)}的意思是\(xy'\)。在OLS估计时还蛮省事。
\item
  \texttt{matlab}包模拟了matlab软件中的许多矩阵函数。
\item
  克罗内克积使用\texttt{\%x\%}或者\texttt{kronecker}。
\item
  \texttt{numDeriv}里面的\texttt{hession}计算海塞矩阵,它的逆的负数,就是极大似然估计的标准差。
\item
  \texttt{Matrix::bdiag(A,A)}生成以两个A为对角元素的分块对角矩阵。
\item
  \texttt{Mod}计算复数的模,不同于\texttt{mod}用来整除。
\item
  \texttt{caracas}: 通过提供对Python SymPy库的访问来实现计算机代数,从而使以符号方式解方程、寻找符号积分、符号和和其他重要量成为可能。
\item
  \texttt{calculus}: 针对数值和符号演算提供了C++优化函数,包括符号算术、张量演算、泰勒级数展开、多元埃尔米特多项式等.
\item
  \texttt{rootSolve}提供了非线性方程(组)的解,以及微分方程的稳态解的形式。
\end{itemize}

\begin{Shaded}
\begin{Highlighting}[]
\NormalTok{fun <-}\StringTok{ }\ControlFlowTok{function}\NormalTok{ (x) }\KeywordTok{cos}\NormalTok{(}\DecValTok{2}\OperatorTok{*}\NormalTok{x)}\OperatorTok{^}\DecValTok{3}
\KeywordTok{curve}\NormalTok{(fun, }\DecValTok{0}\NormalTok{, }\DecValTok{8}\NormalTok{) }\CommentTok{# 先画个图看看解大约在哪里}
\KeywordTok{abline}\NormalTok{(}\DataTypeTok{h =} \DecValTok{0}\NormalTok{, }\DataTypeTok{lty =} \DecValTok{3}\NormalTok{) }\CommentTok{# 把0轴搞出来}
\NormalTok{uni <-}\StringTok{ }\KeywordTok{uniroot}\NormalTok{(fun, }\KeywordTok{c}\NormalTok{(}\DecValTok{0}\NormalTok{, }\DecValTok{8}\NormalTok{))}\OperatorTok{$}\NormalTok{root }\CommentTok{# 此时求解}
\end{Highlighting}
\end{Shaded}

方程组的根求解用\texttt{multiroot}。

\hypertarget{ux6570ux503cux4f18ux5316}{%
\section{数值优化}\label{ux6570ux503cux4f18ux5316}}

\begin{itemize}
\tightlist
\item
  \textbf{无约束优化}。\texttt{optimx}一个优化包,经常用它的\texttt{optimx}函数。但是它是无约束优化,尽管可以包含上下界的约束(盒子约束)。
\end{itemize}

\begin{Shaded}
\begin{Highlighting}[]
\NormalTok{para <-}\StringTok{ }\KeywordTok{list}\NormalTok{(}\DataTypeTok{R1 =} \FloatTok{0.1}\NormalTok{, }\DataTypeTok{phi1 =} \FloatTok{0.8}\NormalTok{, }\DataTypeTok{H1 =} \FloatTok{0.3}\NormalTok{, }\DataTypeTok{A1 =} \FloatTok{0.8}\NormalTok{, }\DataTypeTok{intcp =} \DecValTok{1}\NormalTok{) }\OperatorTok\StringTok{ }\KeywordTok{as.numeric}\NormalTok{()}
\CommentTok{# par就是要优化的参数的初值}
\CommentTok{# fn就是要优化的函数,譬如似然函数,这个函数可以包含多个参数,}
\CommentTok{#    如这里的Y,X,r等。然后那个没写进来的参数就是要优化的参数。}
\CommentTok{#    lnlik <- function(Y = Y, X = X,para = para, r = 1)\{...\}}
\CommentTok{# gr和hess如果必要,可以包含进来。}
\CommentTok{# lower,upper就是搜寻的上下界。}
\NormalTok{a <-}\StringTok{ }\KeywordTok{optimx}\NormalTok{(}\DataTypeTok{par =}\NormalTok{ para, }\DataTypeTok{fn =}\NormalTok{ lnlik, }\DataTypeTok{Y =}\NormalTok{ Y, }\DataTypeTok{X =}\NormalTok{ X, }\DataTypeTok{r =} \DecValTok{1}\NormalTok{, }
            \DataTypeTok{lower =} \KeywordTok{c}\NormalTok{(}\FloatTok{0.01}\NormalTok{,}\FloatTok{0.01}\NormalTok{,}\FloatTok{0.01}\NormalTok{,}\FloatTok{0.01}\NormalTok{,}\OperatorTok{-}\OtherTok{Inf}\NormalTok{), }
            \DataTypeTok{upper =} \KeywordTok{c}\NormalTok{(}\OtherTok{Inf}\NormalTok{, }\FloatTok{0.99}\NormalTok{,}\OtherTok{Inf}\NormalTok{,}\OtherTok{Inf}\NormalTok{,}\OtherTok{Inf}\NormalTok{),}\DataTypeTok{control =} \KeywordTok{list}\NormalTok{(}\DataTypeTok{all.methods =}\NormalTok{ T))}
\KeywordTok{gHgen}\NormalTok{(par, fn) }\CommentTok{# 创造得分矩阵,海塞矩阵。}
\end{Highlighting}
\end{Shaded}

\begin{itemize}
\tightlist
\item
  \textbf{有约束优化}。\texttt{stats::constrOptim}和\texttt{alabama::constrOptim.nl}都可进行有约束的优化,后者是对前者的一个强化,不仅在算法上更牛逼,也可以放入非线性约束。后者是用函数如\texttt{hin(x)\textless{}=0}的形式来表达约束,需要理解的是,约束函数中的参数一定要与目标函数的参数一致,即便约束函数没有用到目标函数的参数。
\end{itemize}

\begin{Shaded}
\begin{Highlighting}[]
\CommentTok{# 比如我的目标函数如下}
\NormalTok{fn <-}\StringTok{ }\ControlFlowTok{function}\NormalTok{ (x,a,b)\{}
\NormalTok{    ...}
\NormalTok{\}}
\CommentTok{# 约束函数如下。一定要把a和b写进去,即便函数中未用到它}
\NormalTok{hin <-}\StringTok{ }\ControlFlowTok{function}\NormalTok{(x, a, b)\{}
\NormalTok{    ...}
\NormalTok{\}}
\CommentTok{# 优化函数如下}
\KeywordTok{constrOptim.nl}\NormalTok{(par, fn, }\DataTypeTok{gr =} \OtherTok{NULL}\NormalTok{, }
\DataTypeTok{hin =} \OtherTok{NULL}\NormalTok{, }\DataTypeTok{hin.jac =} \OtherTok{NULL}\NormalTok{, }\DataTypeTok{heq =} \OtherTok{NULL}\NormalTok{, }\DataTypeTok{heq.jac =} \OtherTok{NULL}\NormalTok{, }
\DataTypeTok{control.outer=}\KeywordTok{list}\NormalTok{(), }\DataTypeTok{control.optim =} \KeywordTok{list}\NormalTok{(), ...)}
\end{Highlighting}
\end{Shaded}

一旦得到优化函数,就可以调用\texttt{numDeriv::hessian(fn,\ par)}来计算海塞矩阵,从而得到标准误之类的。

\hypertarget{ML}{%
\chapter{机器学习和微观计量}\label{ML}}

\hypertarget{ux4e00ux53e5ux8bddtips-7}{%
\section{一句话Tips}\label{ux4e00ux53e5ux8bddtips-7}}

\begin{itemize}
\tightlist
\item
  \texttt{gfoRmula}:一个处理时变处理干扰的R包。
\item
  \texttt{oem}包可以执行各类lasso,group lasso等算法。
\item
  \texttt{gsynth}广义合成控制包。考虑了交互固定效应。
\item
  \texttt{ArCo}: 一个人工反事实的包
\item
  \texttt{ForecastComb}一个集成预测的包,包括stacking 方法。
\item
  \texttt{htree}基于历史回归树,所以可以用于面板数据的随机森林模型。
\item
  \texttt{tsensembler}一个对机器学习多种预测方法进行集成的包,包括随机森林,装袋,支持向量回归等。
\item
  \texttt{knerlab}支持向量机和支持向量回归,他的vig写得好。
\item
  \texttt{grf},一个因果树的包。
\item
  \texttt{spikeslab},一个选择变量的贝叶斯方法的包。
\item
  \texttt{orf}: 实现Lechner和Okasa(2019)中开发的有序森林估计量,以估计具有有序分类结果的模型(有序选择模型)的条件概率.
\item
  \texttt{MatchIt}做匹配感觉很好。主要函数是\texttt{matchit},针对这个函数得到的类,使用\texttt{summay}可以得到匹配前后均值变化,经验分位变化等。这个函数返回的类包含有一个元素\texttt{match.matrix},里面有被匹配的控制组信息。
\end{itemize}

\hypertarget{rocux66f2ux7ebfux7ed8ux5236}{%
\section{ROC曲线绘制}\label{rocux66f2ux7ebfux7ed8ux5236}}

ROC曲线可以同时展现所有可能阈值出现的两类错误。其横轴为1-特异度(1-specificity),纵轴为灵敏度(sensitivity).

调用\texttt{pROC}包即可计算。该包的参数如下:

\begin{Shaded}
\begin{Highlighting}[]
\KeywordTok{roc}\NormalTok{(response, predictor, controls, cases,}
\NormalTok{density.controls, density.cases,}
\DataTypeTok{levels=}\NormalTok{base}\OperatorTok{::}\KeywordTok{levels}\NormalTok{(}\KeywordTok{as.factor}\NormalTok{(response)), }\DataTypeTok{percent=}\OtherTok{FALSE}\NormalTok{, }\DataTypeTok{na.rm=}\OtherTok{TRUE}\NormalTok{,}
\DataTypeTok{direction=}\KeywordTok{c}\NormalTok{(}\StringTok{"auto"}\NormalTok{, }\StringTok{"<"}\NormalTok{, }\StringTok{">"}\NormalTok{), }\DataTypeTok{algorithm =} \DecValTok{5}\NormalTok{, }\DataTypeTok{quiet =} \OtherTok{TRUE}\NormalTok{, }
\DataTypeTok{smooth=}\OtherTok{FALSE}\NormalTok{, }\DataTypeTok{auc=}\OtherTok{TRUE}\NormalTok{, }\DataTypeTok{ci=}\OtherTok{FALSE}\NormalTok{, }\DataTypeTok{plot=}\OtherTok{FALSE}\NormalTok{, }\DataTypeTok{smooth.method=}\StringTok{"binormal"}\NormalTok{,}
\DataTypeTok{ci.method=}\OtherTok{NULL}\NormalTok{, }\DataTypeTok{density=}\OtherTok{NULL}\NormalTok{, ...)}
\end{Highlighting}
\end{Shaded}

\begin{itemize}
\tightlist
\item
  \texttt{response}: 原始的\(y\);
\item
  \texttt{pridictor}: 估计的\(\hat y\);
\item
  \texttt{smooth}:是否平滑ROC曲线;
\item
  \texttt{percent}:是否百分比的形式显示相关信息,如AUC;
\end{itemize}

\hypertarget{randomforestsrcux5305ux4f7fux7528ux62a5ux544a}{%
\section{\texorpdfstring{\texttt{randomForestSRC}包使用报告}{randomForestSRC包使用报告}}\label{randomforestsrcux5305ux4f7fux7528ux62a5ux544a}}

这个包有几大特征,我就说我用过的几个:

\begin{itemize}
\tightlist
\item
  可以因变量多变量建模
\item
  重要性抽样有置信区间
\item
  偏效应计算接口更友好
\end{itemize}

\hypertarget{otherSF}{%
\chapter{与其他软件的交互}\label{otherSF}}

\hypertarget{ux4e00ux53e5ux8bddtips-8}{%
\section{一句话Tips}\label{ux4e00ux53e5ux8bddtips-8}}

\begin{itemize}
\tightlist
\item
  \texttt{stargazer}的一个模版调用:
\end{itemize}

\begin{Shaded}
\begin{Highlighting}[]
\KeywordTok{stargazer}\NormalTok{(regression,}\DataTypeTok{type =} \StringTok{'text'}\NormalTok{,}\DataTypeTok{out =} \StringTok{"../PicTab/cmp.html"}\NormalTok{,}\DataTypeTok{no.space =}\NormalTok{ T,}\DataTypeTok{report =} \KeywordTok{c}\NormalTok{(}\StringTok{'vcp'}\NormalTok{))}
\end{Highlighting}
\end{Shaded}

\texttt{report}意味着报告变量、系数与p值。

\begin{itemize}
\tightlist
\item
  \texttt{officer}: 与微软软件互动的一个包
\item
  \texttt{readstata13}包读Stata13以后的数据格式。
\item
  R语言调用stata
\end{itemize}

用RStata包可以从R里面调用stata,不过要先用\texttt{chooseBinStata()}先设置stata的安装路径。 也可以在R的启动环境中进行配置。
注意在启动环境中(即Rprofile.site文件中)配置时,应增加如下一行,
\texttt{options(RStata.StataPath\ =\ "\textbackslash{}"D:\textbackslash{}\textbackslash{}Program\ Files\ (x86)\textbackslash{}\textbackslash{}Stata14\textbackslash{}\textbackslash{}StataMP-64\textbackslash{}"")}

\hypertarget{rux8bedux8a00ux8c03ux7528matlab}{%
\section{R语言调用Matlab}\label{rux8bedux8a00ux8c03ux7528matlab}}

Matlab里面的三维画图比R要省事很多。这里探讨一下如何从R调用Matlab的一般步骤。

\begin{verbatim}
- 安装R.matlab包。使用`writemat(filename,A=A,B=B)`把R里面的数据写进Matlab并保存成`.mat`格式。
- 安装matlabr包。使用`run_matlab_script`命令来执行一个`.m`脚本。或者使用`R.matlab`包里面的`evaluate`来一个一个地执行matlab命令。或者类似于调用stata:
\end{verbatim}

\begin{Shaded}
\begin{Highlighting}[]
    \KeywordTok{library}\NormalTok{(matlabr)}
\NormalTok{    MatlabCode <-}\StringTok{ '}
\StringTok{    a = 3;}
\StringTok{    b = a+1;}
\StringTok{    '}
    \KeywordTok{run_matlab_code}\NormalTok{(MatlabCode)}
\end{Highlighting}
\end{Shaded}

\begin{itemize}
\tightlist
\item
  R语言读取SPSS(中文字符)
\end{itemize}

\begin{Shaded}
\begin{Highlighting}[]
\CommentTok{# 读英文字符}
\KeywordTok{library}\NormalTok{(foreign)  }
\NormalTok{mydata=}\KeywordTok{read.spss}\NormalTok{(}\StringTok{"data.sav"}\NormalTok{)  }
\CommentTok{# 或者如下}
\KeywordTok{library}\NormalTok{(Hmisc)  }
\NormalTok{data=}\KeywordTok{spss.get}\NormalTok{(}\StringTok{"data.sav"}\NormalTok{) }

\CommentTok{# 读中文字符}
\KeywordTok{library}\NormalTok{(memisc)}
\NormalTok{data1 =}\StringTok{ }\KeywordTok{as.data.set}\NormalTok{(}\KeywordTok{spss.system.file}\NormalTok{(}\StringTok{"data.sav"}\NormalTok{))}
\NormalTok{data =}\StringTok{ }\KeywordTok{as.data.frame}\NormalTok{(data1)}
\end{Highlighting}
\end{Shaded}

\hypertarget{rux4e0epythonux7684ux65e0ux7f1dux5bf9ux63a5}{%
\section{R与Python的无缝对接}\label{rux4e0epythonux7684ux65e0ux7f1dux5bf9ux63a5}}

\begin{itemize}
\tightlist
\item
  第一步,首先配置好环境
\end{itemize}

\begin{Shaded}
\begin{Highlighting}[]
\KeywordTok{library}\NormalTok{(reticulate)}
\KeywordTok{use_condaenv}\NormalTok{(}\StringTok{"D:/Anaconda3"}\NormalTok{)}

\CommentTok{# 安装的python版本环境查看,显示anaconda和numpy的详细信息。放在}
\CommentTok{# use_condaenv()后,以使配置生效}
\KeywordTok{py_config}\NormalTok{()}

\KeywordTok{py_available}\NormalTok{()}\CommentTok{#[1] TRUE   #检查您的系统是否安装过Python}
\KeywordTok{py_module_available}\NormalTok{(}\StringTok{"pandas"}\NormalTok{)}\CommentTok{#检查“pandas”是否安装}
\end{Highlighting}
\end{Shaded}

\begin{itemize}
\tightlist
\item
  第二步,调用有多种方法。我最喜欢这种,就是直接导入python模块,然后用R的风格来调用。此时R里面的美元符号\$相当于python里面的``.''符号 ,如,
\end{itemize}

\begin{Shaded}
\begin{Highlighting}[]
\NormalTok{os <-}\StringTok{ }\KeywordTok{import}\NormalTok{(}\StringTok{"os"}\NormalTok{)}
\NormalTok{os}\OperatorTok{$}\KeywordTok{getcwd}\NormalTok{()}
\NormalTok{os}\OperatorTok{$}\KeywordTok{listdir}\NormalTok{()}\CommentTok{#您可以使用os包中的listdir()函数来查看工作目录中的所有文件}
 
\NormalTok{numpy <-}\StringTok{ }\KeywordTok{import}\NormalTok{(}\StringTok{"numpy"}\NormalTok{)}
\NormalTok{y <-}\StringTok{ }\KeywordTok{array}\NormalTok{(}\DecValTok{1}\OperatorTok{:}\DecValTok{4}\NormalTok{, }\KeywordTok{c}\NormalTok{(}\DecValTok{2}\NormalTok{, }\DecValTok{2}\NormalTok{))}
\NormalTok{y}
\NormalTok{x <-}\StringTok{ }\NormalTok{numpy}\OperatorTok{$}\KeywordTok{array}\NormalTok{(y)}
\NormalTok{x}
\NormalTok{numpy}\OperatorTok{$}\KeywordTok{transpose}\NormalTok{(x)}\CommentTok{#将数组进行转置}
\NormalTok{numpy}\OperatorTok{$}\NormalTok{linalg}\OperatorTok{$}\KeywordTok{eig}\NormalTok{(x)}\CommentTok{#求特征根和特征向量}
\end{Highlighting}
\end{Shaded}

\hypertarget{ux5176ux4ed6}{%
\subsection{其他:}\label{ux5176ux4ed6}}

\begin{itemize}
\tightlist
\item
  当你发现有些包没有,需要安装的时候,可以如下,
\end{itemize}

\begin{Shaded}
\begin{Highlighting}[]
\KeywordTok{library}\NormalTok{(reticulate)}

\CommentTok{# create a new environment }
\KeywordTok{conda_create}\NormalTok{(}\StringTok{"r-reticulate"}\NormalTok{)}

\CommentTok{# install SciPy}
\KeywordTok{conda_install}\NormalTok{(}\StringTok{"r-reticulate"}\NormalTok{, }\StringTok{"scipy"}\NormalTok{)}

\CommentTok{# import SciPy (it will be automatically discovered in "r-reticulate")}
\NormalTok{scipy <-}\StringTok{ }\KeywordTok{import}\NormalTok{(}\StringTok{"scipy"}\NormalTok{)}
\end{Highlighting}
\end{Shaded}

\begin{itemize}
\tightlist
\item
  这是调用时通常需要的代码:
\end{itemize}

\begin{Shaded}
\begin{Highlighting}[]
\KeywordTok{library}\NormalTok{(reticulate)}
\CommentTok{# 可以查你有几个版本的python}
\KeywordTok{py_config}\NormalTok{()}
\CommentTok{# 想使用哪个版本的python}
\KeywordTok{use_python}\NormalTok{(}\StringTok{'C:/Users/sheng/AppData/Local/Continuum/anaconda3/python.exe'}\NormalTok{)}
\CommentTok{# 检查python可不可用}
\KeywordTok{py_available}\NormalTok{()}
\CommentTok{# 检查模块可不可用}
\KeywordTok{py_module_available}\NormalTok{(}\StringTok{'tushare'}\NormalTok{)}
\end{Highlighting}
\end{Shaded}

\hypertarget{ux4e0ejuliaux7684ux5bf9ux63a5}{%
\section{与julia的对接}\label{ux4e0ejuliaux7684ux5bf9ux63a5}}

\hypertarget{juliacallux5305}{%
\subsection{\texorpdfstring{\texttt{JuliaCall}包}{JuliaCall包}}\label{juliacallux5305}}

感觉此包没有类似\texttt{reticulate}包调用python那么无缝。

\begin{itemize}
\tightlist
\item
  在R中执行julia脚本
\end{itemize}

\begin{Shaded}
\begin{Highlighting}[]
\KeywordTok{library}\NormalTok{(JuliaCall)}
\CommentTok{# 设置存放julia二进制文件的文件夹}
\KeywordTok{julia_setup}\NormalTok{(}\DataTypeTok{JULIA_HOME =} \StringTok{'D:/Program Files/Julia-1.4.2/bin'}\NormalTok{)}
\CommentTok{# 几种主要的调用方式,我把我喜欢的写出来的}
\NormalTok{julia}\OperatorTok{$}\KeywordTok{command}\NormalTok{(}\StringTok{"a = sqrt(2);"}\NormalTok{) }\CommentTok{# 在julia环境中产生了变量a}
\NormalTok{ans <-}\StringTok{ }\NormalTok{julia}\OperatorTok{$}\KeywordTok{eval}\NormalTok{(}\StringTok{"a"}\NormalTok{) }\CommentTok{# 把变量a的值传给R环境中的ans变量}
\CommentTok{# 其他的调用方式}
\KeywordTok{julia_command}\NormalTok{(}\StringTok{"a = sqrt(2);"}\NormalTok{)}
\KeywordTok{julia_eval}\NormalTok{(}\StringTok{"a"}\NormalTok{)}
\CommentTok{#> [1] 1.414214}
\DecValTok{2} \OperatorTok\StringTok{ }\NormalTok{sqrt}
\CommentTok{#> [1] 1.414214}
\end{Highlighting}
\end{Shaded}

\begin{itemize}
\tightlist
\item
  R与julia互传变量:前面提到的\texttt{julia\_eval}可以把julia中的变量传出来,使用\texttt{julia\_assign}可以把R中的变量传到julia中去。
\end{itemize}

\begin{Shaded}
\begin{Highlighting}[]
\KeywordTok{julia_assign}\NormalTok{(}\StringTok{'a'}\NormalTok{,}\DecValTok{1}\OperatorTok{:}\DecValTok{5}\NormalTok{)}
\KeywordTok{julia_command}\NormalTok{(}\StringTok{'a'}\NormalTok{)}
\end{Highlighting}
\end{Shaded}

\begin{itemize}
\tightlist
\item
  julia控制台,而且只要你前期\texttt{julia\_setup()}了,这个控制台里面就包含了前期运算时的变量
\end{itemize}

\begin{Shaded}
\begin{Highlighting}[]
\KeywordTok{julia_console}\NormalTok{()}
\CommentTok{# 输入exit 可以退出}
\CommentTok{# julia> exit}
\end{Highlighting}
\end{Shaded}

\begin{itemize}
\tightlist
\item
  它的函数调用非常吸引人:你甚至可以用R对象作为julia函数的参数
\end{itemize}

\begin{Shaded}
\begin{Highlighting}[]
\KeywordTok{julia_install_package_if_needed}\NormalTok{(}\StringTok{"Optim"}\NormalTok{)}
\NormalTok{  opt <-}\StringTok{ }\KeywordTok{julia_pkg_import}\NormalTok{(}\StringTok{"Optim"}\NormalTok{,}
                           \DataTypeTok{func_list =} \KeywordTok{c}\NormalTok{(}\StringTok{"optimize"}\NormalTok{, }\StringTok{"BFGS"}\NormalTok{))}
\NormalTok{rosenbrock <-}\StringTok{ }\ControlFlowTok{function}\NormalTok{(x) (}\FloatTok{1.0} \OperatorTok{-}\StringTok{ }\NormalTok{x[}\DecValTok{1}\NormalTok{])}\OperatorTok{^}\DecValTok{2} \OperatorTok{+}\StringTok{ }\FloatTok{100.0} \OperatorTok{*}\StringTok{ }\NormalTok{(x[}\DecValTok{2}\NormalTok{] }\OperatorTok{-}\StringTok{ }\NormalTok{x[}\DecValTok{1}\NormalTok{]}\OperatorTok{^}\DecValTok{2}\NormalTok{)}\OperatorTok{^}\DecValTok{2}
\NormalTok{result <-}\StringTok{ }\NormalTok{opt}\OperatorTok{$}\KeywordTok{optimize}\NormalTok{(rosenbrock, }\KeywordTok{rep}\NormalTok{(}\DecValTok{0}\NormalTok{,}\DecValTok{2}\NormalTok{), opt}\OperatorTok{$}\KeywordTok{BFGS}\NormalTok{())}
\NormalTok{result}
\end{Highlighting}
\end{Shaded}

\hypertarget{ux5176ux4ed6-1}{%
\chapter{其他}\label{ux5176ux4ed6-1}}

\begin{itemize}
\tightlist
\item
  \texttt{geosphere::distm}根据经纬度算距离。
\item
  \texttt{stats19}一个下载英国交通事故记录的数据库。
\item
  \texttt{pacman}包管理包,如\texttt{p\_load()}。
\item
  \texttt{vitae}: 提供多个模板和功能以简化简历的制作和维护.
\item
  \texttt{DiagrammeR}:R语言画流程图,网络图,太棒了。教程网址:\url{http://rich-iannone.github.io/DiagrammeR/}
\item
  \texttt{progress}包可以显示进度条。更具体的例子见收藏或者帮助。
\end{itemize}

\begin{Shaded}
\begin{Highlighting}[]
\KeywordTok{library}\NormalTok{(progress)}
\NormalTok{pb <-}\StringTok{ }\NormalTok{progress_bar}\OperatorTok{$}\KeywordTok{new}\NormalTok{(}\DataTypeTok{total =} \DecValTok{500}\NormalTok{) }\CommentTok{# 循环中的次数}
\ControlFlowTok{for}\NormalTok{ (i }\ControlFlowTok{in} \DecValTok{1}\OperatorTok{:}\DecValTok{500}\NormalTok{)\{}
\NormalTok{    pb}\OperatorTok{$}\KeywordTok{tick}\NormalTok{()}
\NormalTok{\}}
\end{Highlighting}
\end{Shaded}

\begin{itemize}
\tightlist
\item
  \texttt{swirl}可以编写交互式教程。
\item
  \texttt{RSelenium}包终于在201901月出来了。
\end{itemize}

  \bibliography{mybib.bib}

\end{document}
